\documentclass{beamer}
\usepackage[utf8]{inputenc}
\usepackage[T1]{fontenc}
\title[Short title]{Long title}
\date[Short date]{Long date}
\author[Short author]{Long author (\texttt{email})}

\usetheme{umich}

\begin{document}

\begin{frame}
\titlepage
\end{frame}

\begin{frame}
\frametitle{Outline}
\tableofcontents
\end{frame}

\section{Introduction}
\begin{frame}
\frametitle{Introduction}
Content for the introduction.
\end{frame}

\section{Main Content}
\begin{frame}
\frametitle{Main Content}
Content for the main section.
\end{frame}

\section{Conclusion}
\begin{frame}
\frametitle{Conclusion}
Content for the conclusion.
\end{frame}


\begin{frame} 
\frametitle{Frame title} 
\framesubtitle{Frame subtitle}
\begin{theorem}
There is no largest prime number. \end{theorem} 
\begin{enumerate} 
\item<1-| alert@1> Suppose $p$ were the largest prime number. 
\item<2-> Let $q$ be the product of the first $p$ numbers. 
\item<3-> Then $q+1$ is not divisible by any of them. 
\item<1-> But $q + 1$ is greater than $1$, thus divisible by some prime
number not in the first $p$ numbers.
\end{enumerate}
\end{frame}

\begin{frame}{Frame title}
\begin{itemize}
\item one
\item two
\end{itemize}
\end{frame}

\end{document}
